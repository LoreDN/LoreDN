\documentclass[a4paper, 10pt]{article}

% package imports
\usepackage[utf8]{inputenc}
\usepackage[T1]{fontenc}
\usepackage[english]{babel}
\usepackage{xcolor}
\usepackage{graphicx}
\usepackage{tikz}
\usepackage[hidelinks]{hyperref}

% page margin settings
\setlength{\oddsidemargin}{0pt}
\setlength{\evensidemargin}{0pt}
\setlength{\topmargin}{0pt}
\setlength{\headheight}{0pt}
\setlength{\textwidth}{16cm}
\setlength{\textheight}{24cm}

% definition of the \cvphoto command
\newcommand{\cvphoto}[4]{

    \begin{tikzpicture}[baseline=(pic.base)]

        \clip (0,0) circle (#3/2);
        \node[inner sep = 0pt] (pic) {
                
            \raisebox{-0.45\height}{\includegraphics[width = #3]{#4}}
                
        };
        \draw[line width = #1, draw = #2] (0,0) circle (#3/2);

    \end{tikzpicture}

}

% definition of the \cvsection command
\newcommand{\cvsection}[1]{

    \vspace{0.5cm} \par \noindent \rule{\textwidth}{0.5pt} \vspace{-0.75cm}
    \section*{#1}

}

% definition of the \cvsubsection environment
\newsavebox{\cvsubbox}
\newenvironment{cvsubsection}[2]{

    \vspace{0.5cm} \par \noindent
    \begin{minipage}{\textwidth}

        \colorbox{#1}{

            \parbox{\textwidth}{\textbf{#2}}
                        
        }
        \begin{lrbox}{\cvsubbox}

            \begin{minipage}{\dimexpr \textwidth - 0.5em \relax}

}{
    
            \end{minipage}

        \end{lrbox} \par \noindent
        \begin{tabular}{ @{} p{0.5em} | p{\dimexpr \textwidth - 0.5em \relax} @{} }
            
            &   \usebox{\cvsubbox} \\
            
        \end{tabular}
        
    \end{minipage} \par

}

% definition of the \cvitem command 
\newlength{\cvlabelwidth}
\newcommand{\cvitem}[2]{

    \par \noindent
    \settowidth{\cvlabelwidth}{\textbf{#1:} }
    \hangindent = \cvlabelwidth
    \hangafter = 1
    \textbf{#1:} #2 \par \vspace{0.5em}
    
}

% start of document
\begin{document}

    % Header
    \begin{tabular}{ @{} c c @{} }
        
        \cvphoto{5pt}{blue!30}{3.5cm}{CV_Photo.jpg} &
        \begin{minipage}{\dimexpr \textwidth - 3.5cm \relax}
            
            \centering
            {\Large \textbf{Lorenzo Di Napoli}} \\[0.5em]
            Birth Date: 03/12/2004 \quad | \quad Address: Piazza Bernini 19, Cormano (MI) \\[0.2em]
            Email: lorenzo\_dinapoli@icloud.com \quad | \quad Tel: \texttt{+39 345 599 2100} \\[0.2em]
            Student of Computer Science and Engineering | \texttt{Politecnico di Milano} \\[0.2em]
            Linkedin: \href{https://www.linkedin.com/in/lorenzo-di-napoli-38108a340}{https://www.linkedin.com/in/lorenzo-di-napoli-38108a340} \\[0.2em]
            Github: \href{https://github.com/LoreDN}{https://github.com/LoreDN} \\[0.2em]

        \end{minipage} \\
        
    \end{tabular} \vspace{0.5cm}

    % Educational Path
    \cvsection{Educational Path}

        \cvitem{2023 - present}{Bachelor's Degree in Computer Science and Engineering, Politecnico di Milano (MI).}
        \cvitem{2018 - 2023}{High School Diploma, in Liceo Scientifico opzione Scienze Applicate, Istituto Salesiano Sant'Ambrogio, Milano (MI) --- grade: 71/100.}

    % Technical Skills
    \cvsection{Technical Skills}
   
        \vspace{-0.5cm}
        \begin{cvsubsection}{blue!20}{Programming Languages}

            \cvitem{VHDL}{studied in university courses, used in university projects.}
            \cvitem{Assembly RISC-V64 / x86\_64}{studied in university courses, used in personal projects.}
            \cvitem{C}{studied in university courses, used in university and personal projects.}
            \cvitem{C++}{self-study starting from university basis of C, used in personal projects.}
            \cvitem{Cuda}{studied in an extra-curricular university course, used in university projects.}            
            \cvitem{Java}{studied in university courses, used in university and personal projects.}
            \cvitem{Python}{studied in university courses, used in university and personal projects.}
        \end{cvsubsection}
        
        \begin{cvsubsection}{blue!20}{Technical and Informatics Knowledge from Curricular University Courses}
        
            \cvitem{Operating Systems (Linux)}{basic knowledge of Linux, process and thread management, paging principles, virtual memory and file systems.}
            \cvitem{Algorithms and Data Structures}{knowledge of data structures such as arrays, heaps, lists, hash tables, trees (BST, RBT, ...) and graphs.
                Implementation of sorting and search algorithms, algorithm design techniques (divide et impera, greedy, dynamic programming).}
            \cvitem{Computer Architecture}{knowledge of Assembly dialect (ISA RISC-V32), CPU operation and pipelining, memory hierarchy.}
            \cvitem{Fundamentals of Communication and Internet}{knowledge of the TCP/IP architecture and main network protocols (HTTP, FTP, DNS, UDP, TCP, IP), IPv4/IPv6 addressing, subnetting.}
        
        \end{cvsubsection}

        \begin{cvsubsection}{blue!20}{Technical and Informatics Knowledge from Extra-Curricular University Courses}

            \cvitem{GPU Architecture (CUDA)}{knowledge of GPU architecture, parallel programming with CUDA, optimization of block/thread structure.}
            \cvitem{FPGA Architecture (VHDL and HLS)}{knowledge of FPGA architecture, hardware design (VHDL) and High-Level Synthesis (HLS) via C/C++.}
            %\cvitem{Python for Data Science and AI}{Python programming applied to Data Science and Artificial Intelligence (in particular Neural Networks).}
    
        \end{cvsubsection}
        
        \begin{cvsubsection}{blue!20}{Tools and Supporting Technologies}

            \cvitem{Git}{knowledge of Git, use for code versioning, basic knowledge of GitHub.}
            \cvitem{Makefile, CMake}{used as tools for advanced compilation of C/C++ projects.}
            \cvitem{Valgrind, Cachegrind, Massif, GDB}{use of support tools for advanced debugging of C/C++ code in Linux.}
            \cvitem{LaTeX, Doxygen}{use of LaTeX (manual/generated via Doxygen) as a tool for writing documentation and reports (such as this CV).}
            \cvitem{Vivado and Vitis}{use of Vivado and Vitis tools for synthesis (VHDL and HLS) and simulation of FPGA architecture projects.}
            \cvitem{Docker, Podman}{creation and management of Containers using Docker/Podman, used as isolated environments and toolchains.}

        \end{cvsubsection}

    % Work Experience and Projects
    \cvsection{Work Experience and Projects}

        \vspace{-0.5cm}
        \begin{cvsubsection}{blue!20}{University Projects}

            \cvitem{Movhex}{development of a program that allows creating and editing a 2D hexagonal map, with the objective of finding the lowest-cost path between two points.
                The project was fully implemented in C, using and adapting knowledge learned in the course \textit{"Algorithms and Data Structures"} at Politecnico di Milano.}
                
        \end{cvsubsection}
            
        \begin{cvsubsection}{blue!20}{Personal Projects}
                
            \cvitem{\underline{\href{https://github.com/LoreDN/Cpp}{LDN Library}}}{personal general-purpose library implemented in C++20.}
            \cvitem{\underline{\href{https://github.com/LoreDN/Neural-Networks}{Neural Networks}}}{from-scratch implementation of different neural networks in Python, focused on clustering various geometric shapes (circles, spirals, etc.).}

        \end{cvsubsection}

        \begin{cvsubsection}{blue!20}{Work Experience}

            \cvitem{Lifeguard}{First aid and defibrillator (BLSD) certification, civil and criminal liability for one's actions and those of swimmers.}
                \begin{itemize}

                    \renewcommand{\labelitemi}{-}
                    \item \textbf{June 2025 - December 2025 [116 hours]:} Summer and winter work experience as Lifeguard Assistant P at the sports center \textit{"M.G.M.Sport Srl"} in Paderno Dugnano (MI).
                    \item \textbf{June 2024 - July 2024 [171 hours]:} Summer work experience as Lifeguard Assistant P at the sports center \textit{"M.G.M.Sport Srl"} in Paderno Dugnano (MI).
                    \item \textbf{July 2022 [36.5 hours]:} Summer work experience as Lifeguard Assistant P at the sports center \textit{"M.G.M. Sport Srl"} in Desio (MB). 
            
                \end{itemize}

        \end{cvsubsection}

    % Language Skills
    \pagebreak
    \cvsection{Language Skills}

        \cvitem{Italian}{Native speaker.}
        \cvitem{English}{Level B2, Grade C (Cambridge FIRST Certificate).}
    
    % Certifications and Documents
    \cvsection{Certifications and Documents {\normalsize ---> \underline{\href{https://github.com/LoreDN/LoreDN/raw/main/CV/Certifications.zip}{DOWNLOAD}}}}

        \cvitem{Driver's License}{Category B driver's license --- [18/11/2023]}
        \cvitem{Cambridge FIRST Certificate}{English language certification, level B2 Grade C --- [19/05/2022]}
        \cvitem{Lifeguard Certification}{award of the Lifeguard Assistant certificate (levels P, IP, MIP) at the \textit{"Federazione Italiana Nuoto" (FIN)} --- [09/11/2022]}
        \cvitem{SFA Certificate}{certificate of participation and completion of the \textit{"Scuola Formazione Animatori MGS" (SFA)} at the association \textit{"Movimento Giovanile Salesiano" (MGS)} --- [17/04/2023]}

\end{document}
